% status: 0
% chapter: TBD

\title{Swagger Service for deploying VMs on OpenStack with Apache Libcloud}


\author{Shagufta Pathan}
\affiliation{%
  \institution{Indiana University}
  \city{Bloomington} 
  \state{IN} 
  \postcode{47408}
}
\email{spathan@iu.edu}

% The default list of authors is too long for headers}
\renewcommand{\shortauthors}{Shagufta}


\begin{abstract}
The primary purpose of this project is two fold. Firstly to write code that
leverages the Apache libcloud library to manage cloud resources provided by
OpenStack. The second part of this project is to expose the wrapper code
mentioned earlier via a Swagger based REST service. It will focus on different
aspects like how to work with OpenStack instances, images, keypair management,
and execution of basic commands on the created instances.
\end{abstract}

\keywords{hid-sp18-516, Swagger, REST, Apache libcloud}

\maketitle

\section{Introduction}
OpenStack is an open-source cloud-computing service and is available for free
that provides virtual servers and resources to customers. It is mostly deployed
as infrastructure-as-a-service~\cite{hid-sp18-516-www-openstack}. Whereas Apache
Libcloud is a python library that allows to interact with several popular cloud
service providers with the help of a unified API to interact with different
cloud services. The compute component of libcloud allows to manage cloud and
virtual servers offered by different providers~\cite{hid-sp18-516-www-libcloud}.
Libcloud helps users to deploy one or more virtual machines on OpenStack and
also allows to run commands on the virtual machines. The motivations behind
using Libcloud is to ensure that the application is portable in terms of the
cloud service provider. 

\section{Resources}
This project aims to provide APIs to some of the following resources:
\begin{itemize}
\item List Images
\item List Nodes
\item List flavor
\item Create Node
\item Deploy Node (including setting up SSH KeyPair)
\item Reboot a node
\item Delete a node 
\item Attach floating IP
\item List floating IP 
\end{itemize}


\begin{acks}

  The authors would like to thank Dr.~Gregor~von~Laszewski for his
  support and suggestions to write this paper.

\end{acks}

\bibliographystyle{ACM-Reference-Format}
\bibliography{report} 

